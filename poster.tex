\PassOptionsToPackage{dvipsnames,table}{xcolor}
\documentclass[final]{beamer}
\mode<presentation> {
  \usefonttheme[onlymath]{serif}
}
\usepackage[absolute,overlay]{textpos}
\usepackage{tangocolors}
\setbeamercolor{headline}{fg=tabutter,bg=ta2aluminium}
\setbeamercolor{separation line}{bg=ta2orange}
\setbeamercolor{title in headline}{fg=ta3skyblue}
\setbeamercolor{author in headline}{fg=ta2skyblue}
\setbeamercolor{institute in headline}{fg=ta2skyblue}

\setbeamercolor{framesubtitle}{fg=ta3orange, bg=ta2gray}
\setbeamercolor{author in head/foot}{fg=ta2orange, bg=black}
\setbeamercolor{title in head/foot}{fg=ta2orange, bg=black}

\setbeamercolor*{normal text}{fg=tachameleon, bg=ta3gray}
\setbeamercolor*{block body}{bg=taaluminium,fg=ta3gray}
\setbeamercolor*{block title}{fg=taorange,bg=ta2gray}
\setbeamerfont{block title}{size=\Large,series=\bf}
\setbeamercolor{upper separation line head}{fg=ta2orange}

\setbeamercolor*{example body}{fg=ta3aluminium,bg=black}
\setbeamercolor*{example text}{fg=ta3aluminium,bg=black}
\setbeamercolor*{example title}{bg=taorange,fg=ta2gray}

\setbeamercolor{alerted text}{fg=ta3gray}

\setbeamercolor{structure}{fg=ta3skyblue}

\setbeamertemplate{navigation symbols}{}

\setbeamertemplate{block begin}{
  \vskip.75ex
  \begin{beamercolorbox}[rounded=true,shadow=true,leftskip=1cm,colsep*=.75ex]{block title}%
    \usebeamerfont*{block title}\insertblocktitle
  \end{beamercolorbox}%
  {\ifbeamercolorempty[bg]{block body}{}{\nointerlineskip\vskip-0.5pt}}%
  \usebeamerfont{block body}%
  \begin{beamercolorbox}[rounded=true,shadow=true,colsep*=.75ex,sep=.75ex,vmode]{block body}%
    \ifbeamercolorempty[bg]{block body}{\vskip-.25ex}{\vskip-.75ex}\vbox{}%
}

\setbeamertemplate{block end}{
  \end{beamercolorbox}
}

\setbeamertemplate{headline}{  
  \leavevmode

  \begin{beamercolorbox}[wd=\paperwidth]{headline}
    \begin{columns}[T]
      \begin{column}{.1\paperwidth}
      \end{column}
      \begin{column}{.8\paperwidth}
        \vskip4ex
        \raggedleft
        \usebeamercolor{title in headline}{\color{fg}\textbf{\LARGE{\inserttitle}}\\[1ex]}
        \usebeamercolor{author in headline}{\color{fg}\Large{\insertauthor}\\[1ex]}
        \usebeamercolor{institute in headline}{\color{fg}\large{\insertinstitute}\\[1ex]}     
      \end{column}
      \begin{column}{.1\paperwidth}
      \end{column}
    \end{columns}
  \end{beamercolorbox}

  \begin{beamercolorbox}[wd=\paperwidth]{lower separation line head}
    \rule{0pt}{2pt}
  \end{beamercolorbox}
}

\setbeamertemplate{footline}[page number]{}

\usepackage[utf8x]{inputenc}
\usepackage[english]{babel}
\usepackage{xcolor}
\usepackage{graphicx}

\usepackage[orientation=portrait,size=a0,scale=1.0]{beamerposter}
\usepackage{algpseudocode}

\setbeamertemplate{caption}[numbered]
\setbeamertemplate{footline}{}
 
\newcommand{\columncaption}[1]{
\begin{center}
  {\huge #1}
\end{center}
}

\author{Anton Korobeynikov, Artem Tarasov}
\title{HammerIT: homopolymer-space Hamming clustering for IonTorrent read error correction}
\institute{St. Petersburg State University, Russia}

\setlength{\TPHorizModule}{\paperwidth}
\setlength{\TPVertModule}{\paperheight}
\begin{document}
\begin{frame}
  \begin{textblock}{0.385}(0.01, 0.05)
    \begin{block}{Introduction}
      FIXME: a few words about Ion Torrent sequencing technology
      \vspace{16em}
    \end{block}
    \begin{block}{IonTorrent error profile}
      Corrected flow signal intensities are available in BAM files
      produced by versions of Ion Torrent Suite prior to 3.4. Called
      homopolymer length is obtained as corrected flow signal
      intensity rounded to the nearest integer.

      We have studied flow signal intensity distributions around
      insertion/deletion sites. File B7-295.bam, downloaded from Ion
      Community website, contained 4.6M~insertions, 5.0M~deletions,
      and 1.5M~mismatches.
      
      Overwhelming majority of errors turned out to be
      insertions/deletions of length~1, occuring when flow signal
      intensity is approximately halfway between two adjacent
      integers.

      \begin{figure}[h!]
        \caption{Flow signal intensities at insertion sites}
        \includegraphics[width=\textwidth]{images/overcalls}
      \end{figure}

      \begin{figure}[h!]
        \caption{Flow signal intensities at deletion sites}
        \includegraphics[width=\textwidth]{images/undercalls}
      \end{figure}

      More detailed analysis of errors in IonTorrent data can be found
      in the article
      ``Shining a Light on Dark Sequencing: Characterising Errors in
      Ion Torrent PGM Data'' \emph{(PLoS~Comput~Biol~9(4))}
      
    \end{block}
    
    \begin{block}{Approach}
      FIXME: brief description
      \vspace{16em}
    \end{block}
  \end{textblock}
  

  \begin{textblock}{0.585}(0.405, 0.05)
    \begin{block}{Results}

      We evaluated HammerIT on 6 publicly available datasets,
      using the same pipeline as the authors of the recently published
      article ``Updating benchtop sequencing performance comparison''
      \emph{(Nature~Biotechnology,~v.~31,~no.~4)}.
      In that article, error rate in four Ion~Torrent datasets has
      been assessed. We used the same data plus two extra datasets from
      314v2~chip, which recently became available on Ion~Community~Portal.
      
      Indel/mismatch error rates were calculated for uniquely mapped
      reads before and after correction. For each dataset, correction
      was done in two ways. In the first setup, trimming was done for 
      read ends that couldn't be corrected due to lack of good k-mers, 
      while in the second one such read ends were preserved in the
      output. Relative change in read coverage after
      correction stayed within 0.4\% in all cases.

      \begin{figure}[h]
        \caption{Error rates before and after correction}
        \includegraphics[width=\textwidth]{images/error_rate}
      \end{figure}

      \begin{figure}[h]
        \caption{Error reduction by read position for Sakai 400bp reads}
        \includegraphics[width=0.5\textwidth]{images/error_rate_vs_offset}
      \end{figure}
    \end{block}
  \end{textblock}

  \begin{textblock}{0.32}(0.013, 0.7)
    \begin{block}{Distance calculation}
      \vspace{37.8em}
    \end{block}
  \end{textblock}

  \begin{textblock}{0.205}(0.345, 0.7)
    \begin{block}{K-mer clustering}
      FIXME: here goes description of clustering algorithm, maybe
      also a plot showing cluster size distribution.
    \end{block}
  \end{textblock}

  \begin{textblock}{0.096}(0.560, 0.7)
    \begin{block}{Typical cluster}
      \begin{table}[h!]
        \scalebox{0.539}{
          \begin{tabular}{|l|r|r|}
            \textbf{16-mer} & \textbf{n} & \textbf{qual.} \\ \hline 
            GTGTACATGTCGATGC & \cellcolor{green!100!red!20!white} 113 & \cellcolor{green!100!red!20!white} 1.00 \\
            GTGTACATGTCGATGT & \cellcolor{green!20!red!20!white} 23 & \cellcolor{green!90!red!20!white} 0.90 \\
            GTGTACATGCGATGCT & \cellcolor{green!6!red!20!white} 7 & \cellcolor{green!80!red!20!white} 0.80 \\
            GTGTACATGTCGTGCT & \cellcolor{green!5!red!20!white} 6 & \cellcolor{green!33!red!20!white} 0.33 \\
            GTGTACATGTCATGCT & \cellcolor{green!5!red!20!white} 6 & \cellcolor{green!31!red!20!white} 0.31 \\
            CTTGTACATGTCGATGC & \cellcolor{green!5!red!20!white} 6 & \cellcolor{green!26!red!20!white} 0.26 \\
            GTGTACATGTCTGCTC & \cellcolor{green!4!red!20!white} 4 & \cellcolor{green!28!red!20!white} 0.28 \\
            GTGTACATGTCGATTGC & \cellcolor{green!4!red!20!white} 4 & \cellcolor{green!12!red!20!white} 0.12 \\
            GTGTACATGTGGATGCT & \cellcolor{green!4!red!20!white} 4 & \cellcolor{green!9!red!20!white} 0.09 \\
            GTGTACATGTCCGATGC & \cellcolor{green!3!red!20!white} 3 & \cellcolor{green!41!red!20!white} 0.41 \\
            GTGTACATGTCGAGCT & \cellcolor{green!3!red!20!white} 3 & \cellcolor{green!14!red!20!white} 0.14 \\
            GTGTACATGTCGAATGC & \cellcolor{green!3!red!20!white} 3 & \cellcolor{green!12!red!20!white} 0.12 \\
            GTGTACATGTCGTGTC & \cellcolor{green!3!red!20!white} 3 & \cellcolor{green!10!red!20!white} 0.10 \\
            GTGTACATGCGATGCA & \cellcolor{green!2!red!20!white} 2 & \cellcolor{green!28!red!20!white} 0.28 \\
            GAGTACACTGTCGCTG & \cellcolor{green!2!red!20!white} 2 & \cellcolor{green!4!red!20!white} 0.04 \\
            GTGTACATGTCGATGA & \cellcolor{green!2!red!20!white} 2 & \cellcolor{green!3!red!20!white} 0.03 \\
            GTGTATACATGTCGTG & \cellcolor{green!1!red!20!white} 1 & \cellcolor{green!37!red!20!white} 0.37 \\
            GTGTACATGCGATGTG & \cellcolor{green!1!red!20!white} 1 & \cellcolor{green!32!red!20!white} 0.32 \\
            GTGTACACTGTCGCTA & \cellcolor{green!1!red!20!white} 1 & \cellcolor{green!25!red!20!white} 0.25 \\
            GTGTACACTGTCCGCTA & \cellcolor{green!1!red!20!white} 1 & \cellcolor{green!13!red!20!white} 0.13 \\
            GTGTACATGTCAATGTC & \cellcolor{green!1!red!20!white} 1 & \cellcolor{green!10!red!20!white} 0.10 \\
            GTGTACATGTCATGCCT & \cellcolor{green!1!red!20!white} 1 & \cellcolor{green!9!red!20!white} 0.09 \\
            CAGTACACTGTCGCTA & \cellcolor{green!1!red!20!white} 1 & \cellcolor{green!8!red!20!white} 0.08 \\
            GTGTACATGCGATGCCT & \cellcolor{green!1!red!20!white} 1 & \cellcolor{green!8!red!20!white} 0.08 \\
            GTGTACATGTCGAGCCG & \cellcolor{green!1!red!20!white} 1 & \cellcolor{green!8!red!20!white} 0.08 \\
            GTGTACATGTCGAGGTG & \cellcolor{green!1!red!20!white} 1 & \cellcolor{green!8!red!20!white} 0.08 \\
            GAGTACACACCACTCTG & \cellcolor{green!1!red!20!white} 1 & \cellcolor{green!6!red!20!white} 0.06 \\
            GTGTACATGTCATGCGG & \cellcolor{green!1!red!20!white} 1 & \cellcolor{green!5!red!20!white} 0.05 \\
            GTGTACATGTCGTGCA & \cellcolor{green!1!red!20!white} 1 & \cellcolor{green!4!red!20!white} 0.04 \\
            GTGTACATGTCGGATGC & \cellcolor{green!1!red!20!white} 1 & \cellcolor{green!4!red!20!white} 0.04 \\
            GTGTACATGCGCTGTC & \cellcolor{green!1!red!20!white} 1 & \cellcolor{green!4!red!20!white} 0.04 \\
            GTGTACATGTCGGATGCC & \cellcolor{green!1!red!20!white} 1 & \cellcolor{green!3!red!20!white} 0.03 \\
            GTGTACATGTCATGTA & \cellcolor{green!1!red!20!white} 1 & \cellcolor{green!3!red!20!white} 0.03 \\
            CAGTACACTGTAGCTA & \cellcolor{green!1!red!20!white} 1 & \cellcolor{green!3!red!20!white} 0.03 \\
            GTGTACAATGTCGTGCT & \cellcolor{green!1!red!20!white} 1 & \cellcolor{green!3!red!20!white} 0.03 \\
            ATGTACGCTGTCGCTA & \cellcolor{green!1!red!20!white} 1 & \cellcolor{green!3!red!20!white} 0.03 \\
            GAGTATACATGTCGTG & \cellcolor{green!1!red!20!white} 1 & \cellcolor{green!2!red!20!white} 0.02 \\
            GTGTACATGTCTGCTA & \cellcolor{green!1!red!20!white} 1 & \cellcolor{green!2!red!20!white} 0.02 \\
            GTGTACATGTCGAATGT & \cellcolor{green!1!red!20!white} 1 & \cellcolor{green!2!red!20!white} 0.02 \\
            GTTGTTACATGTCGATGC & \cellcolor{green!1!red!20!white} 1 & \cellcolor{green!2!red!20!white} 0.02 \\
            GTGTATACATGTCATG & \cellcolor{green!1!red!20!white} 1 & \cellcolor{green!1!red!20!white} 0.01 \\
            CAGTACACTGTCGCTG & \cellcolor{green!1!red!20!white} 1 & \cellcolor{green!1!red!20!white} 0.01 \\
            GTGTACATGTCGACTA & \cellcolor{green!1!red!20!white} 1 & \cellcolor{green!1!red!20!white} 0.01 \\
            GTGTACATGTCGATTCT & \cellcolor{green!1!red!20!white} 1 & \cellcolor{green!1!red!20!white} 0.01 \\
            GTGTACATGTCGAGTA & \cellcolor{green!1!red!20!white} 1 & \cellcolor{green!1!red!20!white} 0.01 \\
            GTGTTACATGTCGATGC & \cellcolor{green!1!red!20!white} 1 & \cellcolor{green!1!red!20!white} 0.01 \\
            ATGTACGCTGTAGCTG & \cellcolor{green!1!red!20!white} 1 & \cellcolor{green!1!red!20!white} 0.01 \\
            ATGTACGCTGTAGCTGG & \cellcolor{green!1!red!20!white} 1 & \cellcolor{green!1!red!20!white} 0.01 \\
            GTGTTACATGTGGATGCT & \cellcolor{green!1!red!20!white} 1 & \cellcolor{green!1!red!20!white} 0.01 \\
            GTGTACATGTCGACAGG & \cellcolor{green!1!red!20!white} 1 & \cellcolor{green!0!red!20!white} 0.00 \\
            \hline\end{tabular}
        }
      \end{table}
    \end{block}
  \end{textblock}
  \begin{textblock}{0.32}(0.666, 0.7)
    \begin{block}{Error correction algorithm}

      \begin{algorithmic}
        \For{\textbf{each} \textit{read}}
      \item \hspace{\algorithmicindent}\emph{(Producing contiguous corrected parts of \emph{read})}
          \For{\textbf{each} homopolymer-space \textit{kmer} from the read}
            \State \textit{center} $\gets$ center of the cluster to which \textit{kmer} belongs;
            \If{\textit{center} quality is more than user-specified threshold \textbf{and}\\
              \hspace{4.0em}\textit{center} bases agree with the previous center}
              \State include \textit{center} into consensus score calculation;
            \Else
              \State yield new corrected part from current consensus;
              \State estimate its position on the read;
              \State reset consensus table and start new part;
            \EndIf
          \EndFor
      \item \hspace{\algorithmicindent}\emph{(Combining corrected parts)}
          \For{\textbf{each} corrected \textit{part}}
            \State trim homopolymer runs with low consensus score from both ends;
            \State align last 8 homopolymer runs against the read sequence;
            \State align first 8 runs of the next part;
            \If{there is a gap on the read between the two parts}
              \State copy read homopolymer runs as is;
            \Else
              \State select homopolymer runs with higher consensus score\\
          \hspace{6.0em}from the intersection of the two parts;
            \EndIf
          \EndFor
      \item \hspace{\algorithmicindent}\emph{(Optionally, attaching uncorrected end)}
           \State align last 8 runs of the last chunk against the read sequence;
           \State append read homopolymer runs after the last aligned run.
        \EndFor
      \end{algorithmic}
  \end{block}
\end{textblock}

\end{frame}
\end{document}